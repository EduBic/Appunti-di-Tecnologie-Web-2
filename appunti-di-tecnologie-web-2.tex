
\documentclass[a4paper,11pt]{report}
\usepackage[T1]{fontenc}
\usepackage[english, italian]{babel}
\usepackage[utf8]{inputenc}
\usepackage[xindy]{imakeidx}
\usepackage{xcolor}
\usepackage{graphicx}
\usepackage{amsmath}
\usepackage{amssymb}
\usepackage{etaremune}
\usepackage{enumitem}
\usepackage[toc,page]{appendix}
\usepackage{verbatimbox}

\usepackage[hidelinks, colorlinks=true]{hyperref}	
\usepackage{bookmark}
\usepackage{caption}
\usepackage{subfig}

\captionsetup{tableposition=top, figureposition=bottom, font=small}
\setcounter{tocdepth}{3}
\setcounter{secnumdepth}{3}


\usepackage{epigraph}


\makeindex


\includeonly{
			sezioni/PrimiPassiNelWeb,%
			sezioni/IlSitoWeb,%
			sezioni/ProblemiDiUsabilita,%
			sezioni/SitiEcommerce,%
			sezioni/IlComportamentoDegliUtenti,%
			sezioni/LaPubblicita,%
			sezioni/LaRicerca,%
			sezioni/LaVisibilita,%
			sezioni/IlNomeGiusto,%
			sezioni/LInformazioneEIlWebSemantico,%
			sezioni/MobileWeb,%
			sezioni/SocialWeb,%
			appendici/AnalisiUsabilitaSito,
			appendici/CookieLaw,%
			%appendici/EyeTrackIII,%
			%appendici/WebSpamTaxonomy,%
			%appendici/LOD,%
			%appendici/ScrivereRelazioneUsabilita,%
			%appendici/AffrontareEsame%
			}		
		
\begin{document}

\title{Lezioni su Tecnologie Web 2}
\author{Eduard Bicego}
\date{14-04-2016}

\maketitle

\input{sezioni/Dedica}

\begin{abstract}
	Il presente documento ha l'intento di raccogliere tutto il materiale presentato al corso di tecnologie web 2 dell'università di Padova tenuto dal professor Massimo Marchiori. Il documento ha lo scopo di dare le informazioni necessari ad ulteriore ricerca e approfondimento, come del resto le lezioni del corso. Nonostante gli sforzi questo documento è lungi dall'essere una guida completa su usabilità, motori di ricerca, web semantico e altro. \par
	
	Ho cercato di raggruppare e strutturare nel modo più logico le informazioni al fine di rendere più facile l'apprendimento cercando di mantenere intatto lo stile e il contenuto come è spiegato in aula. \par
	
	Il seguente documento non si prefigge lo scopo di contenere lo stretto indispensabile per l'esame e tanto meno di contenerne tutto il necessario. 
	È usufruibile a tutti nella speranza di essere materiale utile e di supporto. Il simbolo asterisco tra parentesi tonde "(*)" significa che la sezione non è stata sviluppata sufficientemente. Qualsiasi \emph{fork} o incremento su tale documento è ben accetto purché rimangano i riferimenti a me e alle risorse citate in riferimenti.
	\begin{flushright}
		Buon viaggio virtuale \\
		\emph{Eduard}
	\end{flushright}
	
	
\end{abstract}

\hypersetup{linkcolor=black}
\tableofcontents
	%\newpage
\listoffigures
	%

\newpage
	\begin{center}
		\emph{``Abbiamo Internet, il web, gli smartphone,\\
		 ma chi li maneggia sono sempre gli stessi,\\
		  siamo sempre noi uomini, \\
		  con le nostre luci e le nostre ombre \\
		  che sfidano ogni tempo e ogni epoca.''\\}
	\end{center}
	\begin{flushright}
		\emph{Massimo Marchiori} - \emph{Meno Internet più Cabernet}	
	\end{flushright}

\newpage

\hypersetup{linkcolor=blue, urlcolor=blue}



\chapter{Primi passi nel Web (*)}
	
	\section{1945 Memex}
		\textbf{Vanner Bush} scrive un articolo scientifico presentando Memex un sistema innovativo dove sono presentate una fotocamera frontale da posizionarsi in testa (pensate a Google glass) e una super-scrivania tecnologica dove poter immettere informazione con un’importante caratteristica innovativa: i trails ovvero organizzando il tutto con indici associativi per unire due cose assieme.
		
		
	\section{1960-68 NLS: onLine System}
		\textbf{Doug Engelbart} realizzò nel 1968 l'NLS (Online system) un vero e proprio sistema innovativo nel campo informatico dove introdusse una struttura gerarchica di informazioni (il principio dell'XML), un misterioso puntino mosso da uno strumento dotato di due ruote: una per l'asse delle ordinate e una per quello delle ascisse: il primo mouse. Inoltre la documentazione presentava contenuto ipertestuale: ogni parola poteva essere cliccata per condursi in altra documentazione inerente alla parola. In ogni documento era possibile collegarsi da più postazioni ed eseguire un editing in tempo reale con visione e audio degli altri collegati.
		
	\section{1960 Xanadus}
		\textbf{Ted Nelson} è l'inventore del termine ipertesto e del sistema Xanodus (sistema creato solo concettualmente, mai realizzato nonostante vari tentativi). Prevedeva la vendita online di documenti con una letteratura connessa anche grazie alla transclusione. Inoltre era la prima ideazione di un universo di documenti interattivo.
		Le novità interessanti proposte erano:
		\begin{itemize}
			\item i micropagamenti: il nuovo contenuto che prendeva info da altri avrebbe pagato i diritti di questi;
			\item qualunque cosa è collegabile ad altro con uno schema di indirizzi che non solo va in una direzione come il web di oggi ma permette di sapere quali indirizzi puntano a quell'informazione.
			\item implementazione di un Versioning system, in modo da tenere traccia di ogni aggiornamento effettuato sulle informazioni.
		\end{itemize}
		
		
		
		
	\subparagraph*{Morale}
			\begin{quote}
				``I sistemi sociali "Open World" devono essere gratis.''
			\end{quote}
		
	\section{1980 Enquire}
		\textbf{Tim Berners-Lee}, dipendente del CERN e inventore del web, scrive un articolo (enquire) dove presenta un sistema che struttura le info raccolte dal CERN.
		
		
	\section{1990 World Wide Web}
		Prima idea di web, è il 1989 e queste sono le caratteristiche che Tim presenta:
		\begin{itemize}
			\item struttura a ipertesto e non ad albero;
			\item accesso da remoto;
			\item sistema omogeneo con i diversi sistemi operativi;
			\item sistema non centralizzato (questo perché cresca da solo);
			\item informazioni pubbliche e private;
			\item analisi di dati (big data analysis).
		\end{itemize}
	
		Nel 1990 propone il progetto (chiamato www) al CERN, il quale viene rifiutato.
		
		Il progetto WWW forniva un modo per cercare l'informazione con keyword (un motore di ricerca) e il primo browser, un editor/server, in questo modo tutti potevano aggiungere e correggere (come le wiki).
		
		Nel 1991 Tim Berners-Lee propone ad una conferenza il proprio sistema che viene rifiutato per la sua troppa semplicità. Sappiamo infatti che il web di Tim non è per niente soffisticato e già anni prima erano presenti tecnologie molto più innovative. Inoltre la GUI ha limitato la sua potenzialità del progetto.
		
		Nonostante questo apre il primo server pubblico che contiene tutta la documentazione sul sistema.
		
		\subparagraph*{Morale}
		\begin{quote}
			``la GUI e l’hardware confondono il vero potenziale di un'idea.''
		\end{quote}
		
		Nel frattempo nascono alternative alcune molto più estese del web.	
		
		
	\section{Le alternative al WWW}
	
		\subsection{1991 Archie}
		Primo vero motore di ricerca al mondo. Nato prima del web. Motore di ricerca su cosa se non c'è il web? In uno spazio formativo FTP file transfert protocol, all'epoca questo era il sistema per scambiarsi dati in cui altri potevano accedere alle directory messe pubbliche. Come trovare cosa ci serve ecco che nasce Archie. L'idea del motore di ricerca quindi è antecedente al web.
		
		\subsection{WAIS}
		\textbf{Wide Air Information Server}: si occupa del problema di trovare informazione. Si accede direttamente in database cercando stringhe. Discreto successo.
		Sparisce però nel 1993 perchè la ditta che l'aveva inventato fallisce (1995).
		
		\subsection{Gopher}
		In pratica era un sistema che presetava del testo con dei numeri. Una versione molto simile al web di Tim Berners Lee. Differenze? Struttura pagina web fissa e più semplice: o è un menu o del testo o un'immagine. Un web ancora più semplice dell'attuale. La tecnologia Gopher ha un vero e proprio boom rispetto al web.
		
		Nel 1992 il web non è considerato dal pubblico, infatti i server web crescono a 26 server contro i migliaia di server di Gopher.
		Nel frattempo nasce Veronica il motore di ricerca di Gopher. Sembra ormai inesorabile la sconfitta del web. Ma nel 1993 poiché Gopher ha così tanto successo l'università del Minesota mette a pagamento la possibilità di aprire un server Gopher, distribuendo il programma a pagamento. 
		
		Nel giro di pochissimi anni l'asse di interesse si sposta sul sistema Web poiché gratuito e pubblico, Gopher scompare nel dimenticatoio.
		Nel 1993 infatti i server web diventano 50 e il traffico di internet nel web diventa lo 0,1\% a Marzo e a Settembre diventa già l'1\%.
		Nascono Browser per il web dove due sopravvivono: Lynx (un browser ancora oggi sopravvissuto poichè è rimasto un browser testuale).
		
		\subparagraph*{Morale}
		\begin{quote}
			``Perché Lynx si è salvato oltre 30 anni? Perché si è scelto come target una nicchia ben definita, ovvero quella delle console ed è sopravvisuto. Un esempio moderno è Opera che a suo tempo ha puntato sul mobile poiché non poteva competere con Microsoft. Questa scommessa ha poi ripagato con risultati.''
		\end{quote}
		
		Altro browser che distrugge la concorrenza è Mosaic. Nonostante sia più lento degli altri ha una caratteristica importantissima e che oggi consideriamo ovvia: il browser era capace di mostrare le immagini nella pagina stessa del sito. Infatti prima la visualizzazione dell'immagine era effettuata in un'altra pagina.
		
		\subparagraph*{Morale}
		\begin{quote}
			``cosa ha veramente sbaragliato la concorrenza di Mosaic? L’aspetto grafico. Esso può contare molto più della qualità.''
		\end{quote}
		
		Nel 1993 nasce il primo sito commerciale sul web e ad ottobre il numero di server quadruplica. Il sistema si fa nome e la stampa si interessa.
		Nel 1994 ci si ritrova per chiedersi cosa fare del web ormai frutto di interesse di molti. Il traffico web al CERN è mille volte quello di 3 anni prima, tutto è in crescita esponenziale. Al che il CERN stupito e impaurito del troppo interesse sul sistema decide di togliere i finanziamenti (WTF!). Il finanziamento futuro arriverà dal MIT consapevole del potenziale della web.
		
		Nasce Netscape IL BROWSER di riferimento che vince su tutta la concorrenza. Perché? Anch'esso introduce una caratteristica innovativa che lo porta al successo: la visualizzazione della pagina è incrementale, ciò che è scaricato è subito visualizzato a video garantendo così l'eliminazione del ritardo per il caricamento dell’intera pagina. Il caricamento sembra quasi istantaneo. 
		
		\subparagraph*{Morale}
		\begin{quote}
			``Il tempo d’attesa degli utenti è fondamentale.''
		\end{quote}
		
		Nel 1995 avviene la guerra dei browser (vedi corso tecnologie web).
		
		\subparagraph*{Morale}
		\begin{quote}
			``Possiamo imparare da cosa è successo in passato e capire parte del nostro presente. Molto del futuro è già stato immaginato.''
		\end{quote}



\include{sezioni/IlSitoWeb}

\include{sezioni/ProblemiDiUsabilita}

\include{sezioni/SitiEcommerce}

\include{sezioni/IlComportamentoDegliUtenti}

\include{sezioni/LaPubblicita}

\include{sezioni/LaRicerca}

\include{sezioni/LaVisibilita}

\include{sezioni/IlNomeGiusto}

\include{sezioni/LInformazioneEIlWebSemantico}

\include{sezioni/MobileWeb}

\include{sezioni/SocialWeb}


\begin{appendices}

\include{appendici/AnalisiUsabilitaSito}

\include{appendici/CookieLaw}

%\include{sezioni/}

%\include{sezioni/}

\end{appendices}

%\newpage
	\input{biblio/biblio}

\end{document}	