
\chapter{Usabilità: Siti E-Commerce}
	Quando un sito ha come scopo la vendita di prodotti si pensa erroneamente che la cosa più importante sia appunto il prodotto. In realtà oltre al \textbf{prodotto} è molto importante anche \textbf{il prezzo}! Entrambi sono fattori fondamentali in un sito \emph{e-commerce}.

	\section{Il prezzo}
		Gli utenti vogliono il prezzo nel modo più semplice possibile ovvero vicino al prodotto. Purtroppo in molti casi questo non avviene richiedendo all'utente sforzo che genera frustrazione e spesso avviene l'\emph{iper-associazione}.
	
		\subsection{Trappola iper-associazione}
			È un problema che si verifica quando il prezzo è tenuto nascosto fino all'ultimo link del prodotto, questo genera conseguenze negative.
			\begin{itemize}
				\item Ulteriore fatica computazionale, per sapere il prezzo è necessario un passaggio in più.
				\item Si perde il beneficio della mappa creata con lo \emph{scanning}.
				\item \emph{Gambling click}: l'utente è costretto a cliccare col dubbio di fare qualcosa di giusto. Si è misurato che in presenta di un \emph{gambling click} il gradimento del sito scende del 40\% e il link è cliccato solo dal 30\% in media.
			\end{itemize}
			Se non conosco i prezzi? Dobbiamo dare quantomeno un \emph{range} approssimato di prezzo qualsiasi, l'importante è che ci sia un prezzo. Nel web non si deve mai mettere alla prova la fiducia di un utente.
	
	\section{La pubblicità}
		
		\subsection{Pubblicità classica e tecniche}
			Nelle pubblicità classiche che troviamo oramai dappertutto lo scopo è impressionare e colpire l'utente per poco tempo. A tal scopo sono state definite alcune tecniche:
			\begin{description}
				\item[Fishing price:] utilizzo di un prezzo esca che non è quello reale.
				\item[Net price:] utilizzo di un prezzo netto e quindi inferiore a quello che l'utente pagherà
			\end{description}
			Nel web si tende a comparare la pubblicità in un sito web come la pubblicità nella realtà e quindi si riusano le tecniche classiche sopra descritte. Vedremo che non è così, il web è un luogo diverso. Ricordiamo la metafora, quando si è in un sito internet è come essere all'interno di un negozio fisico.
		
		\subsection{Noi e la pubblicità}
			Il nostro cervello è dotato di una memoria a breve termine e una a lungo termine. Il flusso di dati proveniente dalla pubblicità classica è principalmente memorizzato in quella a breve termine, solo una piccola parte è salvata in quella a lungo termine. Quest'ultime informazioni sono il vero e proprio succo della pubblicità: un'idea vaga senza dettagli. Proprio questa sensazione vaga induce l'utente a venire in negozio o a comprare quel prodotto piuttosto di quell'altro.
		
		\subsection{La pubblicità nel web}
			Nel web non si può ricorrere all'idea vaga nella memoria a breve termine, si è lì nel sito, il prezzo è lì segnato e si ricorda. Non è possibile alterare quindi il prezzo per invogliare poi all'acquisto, il periodo temporale è troppo breve. Utilizzare i trucchi della normale pubblicità 		 quindi porta ad innervosire l'utente.
			\begin{description}
				\item[Fishing prize nel web:] 9 utenti su 10 abbandonano il sito, l'1 che resta ha un calo di \emph{trust} del 50\% (meno gradimento e timer ridotti).
				\item[Net price nel web:] l'85\% degli utenti abbandona il sito. 
			\end{description}
			Un esempio di \emph{net price} è quello di non includere le spese di trasporto o assicurazione solo alla fine, l'utente vedrà i soldi aumentare appena prima dell'acquisto creando insoddisfazione e portandolo addirittura ad annullare l'acquisto. La soluzione è quella di usare un carrello dove mostrare ben in vista le spese senza però richiedere i dati personali fino all'effetivo pagamento.
			Si ricorda infine la parola magica \textbf{gratis} o \emph{free} che è un vero e proprio attivatore di sensazioni positive.
			
	\section{Il prodotto}
	
		\subsection{Descrizione del prodotto}
			È sempre richiesta e deve essere in un linguaggio il più possibile semplice e chiaro in modo che sia comprensibile per tutti gli utenti. Mai assumere che l'utente sappia cos'è il prodotto e gli interessi soltanto il prezzo, l'utente si aspetta sempre una descrizione completa altrimenti il sito dà l'impressione di non essere professionale e questo può portare in taluni casi all'abbandono del sito.
			Oltre alla completezza se le descrizioni sono mal fatte queste portano alla migrazione dell'utente nel 99\% dei casi anche dove il prezzo sia più basso gli utenti sceglieranno quello più caro. Se il nostro prezzo è il 20\% in meno solo il 5\% degli utenti ritornerà a comprare nel nostro sito. La descrizione va fatta con cura, da essa l'utente basa l'acquisto per una questione di \emph{trust}, il prezzo competitivo quindi non basta.
		
		\subsection{L'aspetto visivo del prodotto}
			Ricordarsi sempre: l'utente vuole vedere il prodotto e richiede sempre una descrizione visiva (immagine) con qualità full-screen. Un'immagine di un prodotto deve essere sempre cliccabile in modo da mostrare più nel dettaglio il prodotto. La visualizzazione nel dettaglio deve essere ad esclusiva scelta dell'utente altrimenti spesso l'utente non è disposto ad aspettare il caricamento. Ricordiamo inoltre che le immagini 2D semplici sono più apprezzate dagli utenti.
		
			Riportiamo il sito di \href{https://www.jcrew.com/it/womens_category/sunglasses.jsp?intcmp=h1_sunglasses}{\emph{www.jcrew.com}} come esempio di sito E-Commerce. Il sito implementa molte caratteristiche spiegate egregiamente mentre in altre pecca clamorosamente.